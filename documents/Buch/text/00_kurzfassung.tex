

Die Diplomarbeit „Fenrir – Zum Schutz von OT-Netzwerken“ beschäftigt sich mit der Absicherung von IT- sowie OT-Netzwerken, wie sie bei kritischer Infrastruktur oder auch Produktionsbetrieben üblich sind. Dazu wird eine Modell-Kläranlage inklusive Steurungstechnik aufgebaut, als auch dessen dahinterliegendes IT-Netzwerk simuliert, welche als Testumgebungen für mögliche Angriffe dienen und in weiterer Folge abgesichert werden.

Wie wurde vorgegangen?
YGGDRASIL: Das Unternehmensnetzwerk besteht aus Cisco Routern, Cisco Switches und Firewalls der Firma Fortinet. Zusätzlich wird ein Active Directory verwendet, in dem sich Windows Server und Windows Clients befinden. Außerdem gibt es einige Serverdienste, die von Linux Systemen bereitgestellt werden.

Was bedeuten deine Ergebnisse?
YGGDRASIL: Für die Sicherheit der Endgeräte sorgt eine Endpoint Detection and Response (EDR) Software. Des Weiteren wird die Netzwerkebene durch ein Network Detection and Response (NDR) System überwacht. Den Kern bildet ein Security Information and Event Management (SIEM), das die Daten aus den Bereichen EDR, NDR, Active Directory und Firewall sammelt. Dort kann eine tiefere Analyse auf Basis aller Meldungen im Netzwerk, durchgeführt werden.

Was wurde herausgefunden?
YGGDRASAIL: Nach der Konfiguration der Analysen wird die Sicherheit des Unternehmensnetzwerks getestet, indem Angriffe auf die verschiedenen Teilbereiche durchgeführt werden. Dadurch soll überprüft werden, ob eine automatisierte Erkennung von Attacken möglich ist.
