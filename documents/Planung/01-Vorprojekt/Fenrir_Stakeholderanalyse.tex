% Gabriel Vogler

\documentclass[
	headings=optiontotocandhead,% Erweiterung für das optionale Argument der
	% Gliederungsbefehle aktiviert.
	oneside,
	numbers=noenddot,% Keine Punkte am Ende der Gliederungsnummern und davon
	% abgeleiteten Nummern
	toc=flat, %Flache TOC --- kann man anpassen (auskommentieren)
	10pt, % Schriftgröße
	parskip=full, % Abstand zwischen Absätzen (ganze Zeile)
	listof=totoc, % Verzeichnisse im Inhaltsverzeichnis aufführen
	listof=flat, % mehr Abstand für grosse Zahlen
	numbers=noenddot, % kein Punkt am Ende bei Nummern
	%%enlargefirstpage,% Gibt es bei scrartcl nicht!!!!
	bibliography=totoc, % Literaturverzeichnis im Inhaltsverzeichnis aufführen
	%index=totoc, % Index im Inhaltsverzeichnis aufführen
	%captions=tableheading, % Beschriftung von Tabellen für Ausgabe oberhalb
	% der Tabelle formatieren
	%draft % Status des Dokuments (final/draft) draft hinzufügen zum anziegen
	%%der zeilen ende
	a4paper,DIV=14,
	% captions=tablesignature,
]{scrartcl}

\setcounter{secnumdepth}{3}

\usepackage[T1]{fontenc}
\usepackage[utf8]{inputenc}

\usepackage[english, ngerman]{babel, varioref} % your native language must be the last one!!

\usepackage{lastpage}
\usepackage{listings}
\usepackage{blindtext}

%% Aufzählungen nicht so weit einrücken
\usepackage[inline]{enumitem}
%\setitemize{leftmargin=*}
% Listen etwas wenige einrücken, erfordert enumitem
\setitemize{labelindent=2em,labelsep=0.5cm,leftmargin=12ex}

\usepackage{lmodern}

\usepackage{xspace}

\usepackage{graphicx}
\graphicspath{ {../../assets} }

%%? \usepackage{textcomp}
\usepackage[hyphens]{url}
\usepackage{makeidx}
\makeindex
%%? \usepackage{graphicx}
\usepackage[numbers]{natbib}
\PassOptionsToPackage{normalem}{ulem}
\usepackage{ulem}

\usepackage{needspace}

\setlength\partopsep{0.5ex}%schoenere Listen
\usepackage[bottom]{footmisc}%fussnote ganz unten

\usepackage[]{microtype}
\UseMicrotypeSet[protrusion]{basicmath} % disable protrusion for tt fonts

\usepackage{multirow}   % Allows table elements to span several rows.
\usepackage{booktabs}   % Improves the typesettings of tables.
\usepackage{subcaption} % Allows the use of subfigures and enables their referencing.
\usepackage[ruled,linesnumbered]{algorithm2e} % Enables the writing of pseudo code.
\usepackage[usenames,dvipsnames,table]{xcolor} % Allows the definition and use of colors. This package has to be included before tikz.
\usepackage{nag}       % Issues warnings when best practices in writing LaTeX documents are violated.
\usepackage{todonotes} % Provides tooltip-like todo notes.

\usepackage{color}
\usepackage[binary-units]{siunitx}

%% Override default figure placement To be within the flow of the text rather
%% than on it's own page.
% \usepackage{float}
% \makeatletter
% \def\fps@figure{H}
% \makeatother

%% bei vielen Bildern o.ä sinnvoll: Seite muss nicht bis ganz unten gefüllt werden
% \raggedbottom

%\usepackage{footbib} %  footcite, needs other tooling
%% for pandoc2 images
\makeatletter
\def\maxwidth{\ifdim\Gin@nat@width>\linewidth\linewidth\else\Gin@nat@width\fi}
\def\maxheight{\ifdim\Gin@nat@height>\textheight\textheight\else\Gin@nat@height\fi}
\makeatother
% Scale images if necessary, so that they will not overflow the page
% margins by default, and it is still possible to overwrite the defaults
% using explicit options in \includegraphics[width, height, ...]{}
\setkeys{Gin}{width=\maxwidth,height=\maxheight,keepaspectratio}

%% bessere Suche im PDF
\input{glyphtounicode}
\pdfgentounicode=1
%%%%%%%%%%%%%%%%%%%%%%%%%%%%%%%%%%%%%%%%%%%%%%%%%%%%%%%%%%%%%%%%%%%%%%%%%%%%%%%%%%

%  Kopf und Fußzeilen -- links und rechts verschieden
\newcommand{\kopfbild}{\voffset7mm\includegraphics[width=25mm]{HTL3RLogo}}
\newcommand{\kopfHTL}{\sffamily{\textbf{\large{Projekthandbuch HTL3R}}}}

\usepackage[automark,footsepline,plainfootsepline]{scrlayer-scrpage}
\setkomafont{pageheadfoot}{\normalcolor\footnotesize\scshape}
\setkomafont{pagenumber}{\normalfont\normalsize}
\clearpairofpagestyles
\ihead{\headmark}
\ihead{\kopfbild}
\ohead{\kopfHTL}
\ifoot{\smaller{Höhere Technische Bundeslehranstalt Wien 3 | Rennweg 89b | 1030 Wien | \textcolor{orange}{www.htl.rennweg.at}}}
\ofoot{Seite \pagemark/\pageref{LastPage}}
\ModifyLayer[addvoffset=-.6ex]{scrheadings.foot.above.line}% Linie verschieben
\ModifyLayer[addvoffset=-.6ex]{plain.scrheadings.foot.above.line}% Linie verschieben
\setlength{\headheight}{32pt}

% alle Seiten mit Kopfzeile
%\renewcommand{\chapterpagestyle}{scrheadings}

%% Code Beispiele
%% eine Variante
\usepackage{listings}
\renewcommand{\lstlistingname}{\inputencoding{utf8}Listing}

\usepackage{tabularx}
\usepackage{tabulary}
\usepackage{scrhack}
\usepackage{longtable}
\usepackage{ragged2e}

\newcommand{\positiv}{\textbf{\textcolor[HTML]{00B050}{positiv}}}
\newcommand{\negativ}{\textbf{\textcolor[HTML]{FF0000}{negativ}}}
\newcommand{\green}[1]{\textcolor[HTML]{00B050}{#1}}
\newcommand{\red}[1]{\textcolor[HTML]{FF0000}{#1}}

\usepackage{array}
\newcommand\Tstrut{\rule{0pt}{3.2ex}}         % = `top' strut
\newcommand\Bstrut{\rule[-1.5ex]{0pt}{0pt}}   % = `bottom' strut

\newenvironment{nstabbing}
	{\setlength{\topsep}{-\parskip}
		\setlength{\partopsep}{-\parskip}
		\tabbing}
	{\endtabbing}
	
\newenvironment{myitemize}{
    \begin{itemize}[leftmargin=1.7em]
}{
    \end{itemize}
}

\usepackage{titlesec}
% \titleformat{?Überschriftenklasse?}[Absatzformatierung?]{?Textformatierung?} {?Nummerierung?}{?Abstand zwischen Nummerierung und Überschriftentext?}{?Code vor der Überschrift?}[?Code nach der Überschrift?]
\titleformat{\section}[hang]{\Large\bfseries\sffamily}{\thesection\quad}{-1.2ex}{}
\titleformat{\subsection}[hang]{\large\bfseries\sffamily}{\thesubsection\quad}{-1.2ex}{}
\titleformat{\subsubsection}[hang]{\large\bfseries\sffamily}{\thesubsubsection\quad}{-1.2ex}{}
\titleformat{\paragraph}[hang]{\large\bfseries\sffamily}{\theparagraph\quad}{-1.2ex}{}

% \titlespacing{?Überschriftenklasse?}{?Linker Einzug?}{?Platz oberhalb?}{?Platz unterhalb?}[?rechter Einzug?]
\titlespacing{\section}{0pt}{6pt}{6pt}
\titlespacing{\subsection}{0pt}{6pt}{0pt}
\titlespacing{\subsubsection}{0pt}{6pt}{0pt}
\titlespacing{\paragraph}{0pt}{6pt}{0pt}

%% sollte das letzte Package sein
\usepackage[unicode=true,
bookmarks=true,bookmarksnumbered=false,bookmarksopen=false,
breaklinks=true,pdfborder={0 0 0},backref=false,colorlinks=false]
{hyperref}
\hypersetup{pdftitle={Fenrir Stakeholderanalyse},
	pdfauthor={Bastian Uhlig},
	pdfsubject={DA},
	pdfkeywords={5CN, Fenrir, DA}}
\urlstyle{same} % don't use monospace font for urls

% Auch Fußnoten bündig ausrichten
\deffootnote[]{1em}{1em}{\textsuperscript{\thefootnotemark\ }}
%% setup
\sloppy % weniger Meldungen
\voffset7mm % etwas nach unten

\newcolumntype{M}[1]{>{\centering\arraybackslash}m{#1}}

%%%%%%%%%%%%%%%%%%%%%%%%%%%%%%%%%%%%%%%%%%%%%%%%%%%%%%%%%%%%%%%%%%%%%%%%%%%%%%%%%%
\begin{document}
%% schöner: 10000 -- gar keine, 1000 als Mittelweg
\clubpenalty = 10000 % Schusterjungen verhindern
\widowpenalty = 10000 % Hurenkinder verhindern
\displaywidowpenalty = 10000

{\sffamily{\textbf{\LARGE{\textcolor{orange}{Stakeholderanalyse}}}}}\\
\noindent\rule{\textwidth}{0.1pt}
\begin{nstabbing}
	\hspace{4cm}\=\hspace{4cm}\=\hspace{4cm}\=\kill
	Projekttitel: \> \textbf{Fenrir}\\
	Auftraggeber: \> \textbf{Christian Schöndorfer}\\
	Auftragnehmer: \> \textbf{David Koch}\\
	Schuljahr: \> \textbf{2024/25}
	\> Klasse: \> \textbf{5CN}\\
\end{nstabbing}
{\smaller
	\begin{tabularx}{\textwidth}{l l l l}
	\hline
	\textbf{Version} & \textbf{Datum} & \textbf{Autorin/Autor} & \textbf{Änderung}\Tstrut  \\
	v1.0 & 13.05.2024 & Gabriel Vogler & Erstellung des Dokuments  
\Tstrut \\
	v1.1 & xx.xx.2024 & Bastian Uhlig & Entfernen der Schöller Network group	
\Bstrut \\
	\hline
	\end{tabularx}
}


\section{Projekt Stakeholder-Liste}
\begin{longtable}{|M{1cm}|M{2.5cm}|M{3cm}|M{2cm}|M{2cm}|M{3.5cm}|}
\hline

\hline
\textbf{\normalsize{Nr. (Ext/Int)}} &
\textbf{\normalsize{Bezeichnung des Stakeholders}} &
\textbf{\normalsize{Beschreibung des Stakeholders}} &
\textbf{\normalsize{Einfluss auf Projekt/Macht
(gering/hoch)}} &
\textbf{\normalsize{Einstellung zum Projekt 
(\positiv / \negativ)}} &
\textbf{\normalsize{Maßnahmen}} \\ \hline
E-01 & Christian Schöndorfer & Auftraggeber & hoch & \positiv &
\RaggedRight{
\begin{myitemize}
	\item Termingerechte Fertigstellung
	\item hohe Qualität der Ergebnisse
	\item Alle 2 Wochen Meetings
	\item jede andere 2.Woche einen Statusbericht
\end{myitemize}} \\ \hline 
E-02 & Clemens Kussbach & Auftraggeber Stv. & hoch & \positiv &
	\begin{myitemize}
	\item Termingerechte Fertigstellung
	\item hohe Qualität der Ergebnisse
	\item Alle 2 Wochen Meetings
	\item jede andere 2.Woche einen Statusbericht
	\end{myitemize} \\ \hline
E-03 & Fortinet & Sponsor & gering & \positiv &
	\begin{myitemize}
	\item Erwähnung in diversen Präsentationen, Webauftritt, Social Media und im Diplomarbeitsbuch
	\end{myitemize} \\ \hline 
E-04 & Ikarus & Sponsor & gering & \positiv &
	\begin{myitemize}
	\item Erwähnung in diversen Präsentationen, Webauftritt, Social Media und im Diplomarbeitsbuch
	\end{myitemize} \\ \hline 
E-05 & HTL Rennweg INKO Verein & Sponsor & niedrig & \positiv &
	\begin{myitemize}
	\item Erwähnung in diversen Präsentationen, Webauftritt, Social Media und im Diplomarbeitsbuch
	\end{myitemize} \\ \hline
E-06 & IT/OT-Security Unternehmen & Konkurrenz & niedrig & \negativ &
	\begin{myitemize}
    \item Marktbeobachtung und Analyse
    \item Entwicklung von Alleinstellungsmerkmalen
	\end{myitemize} \\ \hline
I-01 & David Koch & Projektleiter & hoch & \positiv & 
	\begin{myitemize}
	\item Schnellstmögliches Informieren zu Änderungen
	\item Befragung bei größeren Problemen
	\end{myitemize} \\ \hline
I-02 & Bastian Uhlig & Projektleiter Stv. & hoch & \positiv & 
	\begin{myitemize}
	\item Schnellstmögliches Informieren zu Änderungen
	\item Befragung bei größeren Problemen
	\end{myitemize} \\ \hline
I-03 & Gabriel Vogler & Projektmitarbeiter & hoch & \positiv & 
	\begin{myitemize}
	\item Schnellstmögliches Informieren zu Änderungen
	\item Befragung bei größeren Problemen
	\end{myitemize} \\ \hline
I-04 & Julian Burger & Projektmitarbeiter & hoch & \positiv & 
	\begin{myitemize}
	\item Schnellstmögliches Informieren zu Änderungen
	\item Befragung bei größeren Problemen
	\end{myitemize} \\ \hline




\end{longtable}

\section{Stakeholder Portfolio}
\begin{table}[h]
\begin{tabularx} {\textwidth} {|X|X|}
\hline

\footnotesize{\textcolor[HTML]{A6A6A6}{Zufriedenstellen}}
& \footnotesize{\textcolor[HTML]{A6A6A6}{Eng managen (hoher Aufwand)}} \\ 
& \green {E-01 Christian Schöndorfer} \\
& \green {E-02 Clemens Kussbach} \\
& \green {I-01 David Koch} \\
& \green {I-02 Bastian Uhlig} \\
& \green {I-03 Gabriel Vogler} \\
& \green {I-04 Julian Burger}
 \\ \hline 

\footnotesize{\textcolor[HTML]{A6A6A6}{Überwachen (wenig Aufwand betreiben)}}
& \footnotesize{\textcolor[HTML]{A6A6A6}{Informieren/auf dem Laufenden halten}} \\
\red {E-06 IT/OT-Security Unternehmen} 
& \green {E-03 Fortinet} \\
& \green {E-04 Ikarus} \\
& \green {E-05 HTL Rennweg INKO Verein}

 
 \\ \hline 



\end{tabularx}
\end{table}

\end{document}